In this thesis, we investigated the use of a new method, LoRE-SD, to derive ODFs from dMRI data. We did so by performing a group analysis to study the changes in white matter induced by chemotherapy in breast cancer patients. LoRE-SD proved to be a suitable method for FBA, offering valuable flexibility in how ODFs are scaled. The results were broadly consistent with those of the state-of-the-art MT-CSD. Importantly, we showed that FBA results depend not only on the ODFs themselves, but also on the way these ODFs are modulated. Specifically, with LoRE-SD, holding the ODF constant while varying the scaling contrast led to different outcomes. This highlights how applying multiple scaling contrasts allows to extract complementary information from the same data, potentially reflecting distinct tissue microstructural features.
\\Prior to the analysis, we hypothesized that chemotherapy-treated patients would show decreased fixel-wise metrics compared to untreated individuals, reflecting a loss of white matter integrity. Contrary to expectations, we observed significant increases in apparent fibre density, notably in the fornix, a tract associated with memory and cognitive functions. This finding may reflect compensatory mechanisms or structural reorganization in response to chemotherapy-induced inflammation, potentially manifesting as an increase in axon number in the fornix. Such neuroplastic changes might be the brain's adaptive response to inflammatory processes triggered by treatment.
However, the fornix and other significant regions are located near the ventricles, making them particularly susceptible to partial volume effects. In addition, because the fornix is a very thin structure, misregistration during spatial normalization could also contribute to the observed differences. \\Although FBA allowed us to detect these effects, interpreting them remains challenging. 
\\Further research is needed both to clarify the biological meaning of the metrics extracted with LoRE-SD and to confirm the robustness of the observed findings.