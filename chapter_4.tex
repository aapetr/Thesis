This chapter examines the application of LoRE-SD to FBA, highlighting how different ODF scaling contrasts affect results. We discuss how LoRE-SD’s flexibility can influence statistical outcomes and potentially provide biologically meaningful metrics beyond MT-CSD. Group differences, including unexpected increases in apparent fibre density in chemotherapy patients, are discussed in the context of compensatory mechanisms, protective responses, and atrophy. Finally, study limitations and future directions are outlined.

\section{LoRE-SD scaling contrasts}
Results in section~\ref{sec:cont} show how LoRE-SD is able to generate contrasts using the response function representation. Figure~\ref{fig:contrast_combined} demonstrates how an ad-hoc definition of the intra-axonal contrast highlights the same basis functions as a data-driven estimation of WM volume fraction. Both show high values for columns corresponding to low $\lambda_{\perp}$, with weights rapidly decreasing as $\lambda_{\perp}$ increases. This similarity supports the use of the intra-axonal contrast, rather than relying on data-driven approaches based on volume fractions estimations.

In this work, we derived a WM contrast matrix by solving an optimization problem separately for each subject, resulting in subject-specific matrices used to scale the corresponding ODFs. A potentially more robust alternative would involve solving a single optimization problem using the data (Gaussian fractions and WM volume fractions from MT-CSD) from all subjects or a representative subset. The resulting matrix could then be used to scale ODFs across all subjects in a consistent manner.
Additionally, such a matrix would capture the characteristics of a WM response function, defining how to weight and combine Gaussian basis functions to represent WM, similar to the study-specific response function used in MT-CSD. Unlike MT-CSD, which requires recomputing the response function for each study, the contrast matrix approach could provide a reusable representation applicable across different FBA studies, independent of the b-values acquired in a given study.
\\Furthermore, the WM contrast matrix derived through optimization shows lower values compared to those manually defined (Figure~\ref{fig:contrast_combined}). In particular, while the other matrices span the full range from 0 to 1, the WM contrast matrix reaches a maximum of around 0.8 across individuals. This results in the ODFs scaled with the contrast derived from this matrix having lower amplitudes than in the other cases. While for MT-CSD and intra-axonal-scaled LoRE-SD, 0.07 was the chosen threshold to extract fixels (Section \ref{sec:extract}), when scaling with the WM contrast the threshold on the lobe amplitude had to be adjusted. To get a number of fixels on the order of 300,000, 0.03 was chosen.  These threshold choices could have had an impact on the results that might need future investigation. An alternative approach could involve normalizing the values of the WM contrast matrix to the range 0-1, which would allow the use of the same amplitude threshold as in the other cases. However, we expect that the results would not differ significantly and the chosen approach is likely comparable.


The most computationally demanding step in the FBA pipeline is the creation of a population template, which requires multiple iterations of image registration. Given the flexibility of LoRE-SD-derived ODFs, which can be scaled in various ways, we propose using a single template based on one contrast, and applying the same subject-to-template transformation for ODFs scaled using other contrasts. This is possible because the different ODFs are simply scaled versions of the same unit-normalized ODFs.
In our implementation, the intra-axonal scaled template was used as the common reference space and the same warps were used to transform images scaled using the anisotropic intra-axonal and WM contrasts.
Fixel-wise metrics were subsequently extracted by integrating ODF lobes. These metrics are analogous to AFD (e.g., intra-axonal-weighted AFD, FA-weighted AFD, etc.). The flexibility in the response function representation allows the derivation of these additional metrics that might have a biological meaning.

FBA was performed using various scaling contrasts for the ODFs. This procedure involves multiple steps and can be time-consuming, but after selecting suitable hyperparameters, the analysis was fully automated. A simpler alternative could involve performing a group comparison directly on the new contrasts obtained for each subject, without scaling the ODFs and conducting FBA. This approach would resemble voxel-based analysis (VBA), where differences are assessed voxel by voxel, but each voxel value would still reflect information from a specific tissue compartment and meaningful differences among subject could be found. Such an approach could be explored in future studies.

\section{Masks}
As mentioned in section~\ref{sec:mask}, a more stringent masking method was required for LoRE-SD compared to MT-CSD. This is because including voxels outside the brain led to highly noisy and disproportionately large SH coefficients in the scaled ODFs. These outliers resulted from regions with near-zero radial and axial diffusivity (e.g., air or skull). Specifically, for the intra-axonal contrast this resulted in really big scaling factors, due to having close to zero radial diffusivity. This was problematic especially during affine registration, as these big values completely guided the transformations leading to implausible shears and scalings. The SynthStrip mask couldn't exclude the outliers for all the subjects, even after erosion was applied. The method that uses tissues volume fractions, although relying on the results of MT-CSD, provides masks able to exclude all noisy ODFs (Figure~\ref{fig:mask}).
Alternative strategies could be used for brain tissue volume fraction estimation, for example using probabilistic models such as Gaussian mixture models \cite{zhang2001}.

\section{Impact of Different Representations}
Even though MT-CSD and LoRE-SD produced generally different results, some degree of overlap was observed. For example, the fornix was identified as a significant region when using MT-CSD-derived ODFs, and also when using LoRE-SD-derived ODFs, but only when the WM contrast (itself derived from MT-CSD) was applied. This suggests that the response function representation of LoRE-SD can be used to mimic the previous method, effectively capturing similar microstructural features.
Additionally, for LoRE-SD the results were sensitive to the choice of contrast scaling. Brain regions identified as significant with one scaling approach were not necessarily detected with others, indicating the flexibility of LoRE-SD. More importantly this shows how the outcome of FBA is affected by ODF modulation, something previously under investigated as only one scaling factor, the apparent WM volume fraction used in MT-CSD, was available.


\section{Chemobrain}
To account for potential confounding from psychological stress due to the diagnosis, a cancer control (CM) group was included. No significant differences were found between CM and HC, suggesting that stress alone is unlikely to cause detectable white matter changes. This supports the hypothesis that cognitive symptoms and structural alterations in chemotherapy patients (CP) may result from chemotherapy-induced neurotoxicity. Unexpectedly, an increase in AFD was observed in the CP group, mainly in regions associated with cognitive function, particularly the fornix.

Increases in fixel-wise metrics have been observed in other neurological conditions and have been interpreted as compensatory reorganization or protective responses to neuroinflammation (\cite{Verhelst2019,Andica2021}). In the fornix, a central pathway connecting the hippocampus with other memory-related regions which plays a key role in cognition and episodic memory recall(\cite{Li2022, Senova2020}), such changes may reflect adaptive remodeling aimed at maintaining cognitive function. 

Despite the increase in AFD, no significant changes were observed in FC or the combined metric FDC. This suggests that microstructural changes, such as altered axonal packing or diameter, may occur without modifying the macroscopic cross-sectional area or overall capacity of the fibre bundle. However the lack of FDC changes weakens the interpretation of a true structural reorganization and highlights the need to consider alternative explanations.

Technical factors may also contribute to the observed effects. The fornix is a very thin structure, approximately 3 mm in width \cite{Yucel2002}, making it particularly vulnerable to partial volume effects from adjacent cerebrospinal fluid as well as to misregistration across subjects. Atrophy of the fornix has been reported in conditions such as Alzheimer’s disease \cite{Ali2025}, and if similar changes occurred in our patients, they could further amplify these issues, producing apparent AFD increases that may not reflect genuine microstructural alterations. Similarly, regions near the ventricles are especially prone to such artifacts, underscoring the need for cautious interpretation.

In summary, the increase in AFD observed in the fornix may result either from true white matter reorganization or protective adaptation in response to chemotherapy-induced neurotoxicity, or from artifacts related to partial volume effects or misregistration. Regardless of the underlying mechanism, the involvement of the fornix is consistent with its critical role in cognition, linking the hippocampus to broader memory networks and potentially explaining the cognitive deficits observed in chemobrain.


\section{Improving Statistical Power by Reducing the Fixel Mask}
As expected, reducing the number of fixels through TDI-based thresholding increased the number of significant fixels in each test, likely due to improved statistical power. A region near the CC was more prominently identified than in the original fixel mask (Figure~\ref{fig:reduced_MT} and~\ref{fig:reduced_intra}). However, even with the reduced fixel mask, no regions were identified with significantly lower fixel-wise metrics in patients compared to controls.
A more targeted alternative could be to restrict the analysis to specific tracts of interest, such as the CC or superior longitudinal fasciculus. 
\\Notably, after reducing the fixel mask, the fornix emerged as a region showing increased AFD in the CP group compared to HC, even when using the intra-axonal contrast for ODF scaling, a result not observed with the original fixel mask.

\section{Limitations of the work}
The mechanisms of chemotherapy-induced WM damage and the spatial distribution of affected regions is not yet fully understood. If, as reported by Schroyen et al. \cite{Schroyen2023}, chemotherapy primary leads to the enlargement of existing lesions rather than the formation of new ones, this may result in spatially heterogeneous abnormalities. Such heterogeneity challenges the assumptions of voxel- or fixel-wise group comparisons, which rely on spatially consistent effects across individuals. This could explain the limited number of significant findings in our analysis, particularly why we found no decreases in fixel metrics in chemotherapy patients compared to healthy controls. Notably, previous studies investigating chemobrain in breast cancer patients using diffusion MRI-derived metrics have also reported inconsistent group-level differences. While some studies have found significant effects using DTI-based measures, group differences using FBA have not been found \cite{Deprez2011, Schroyen2021}.
\\Statistical analysis using Connectivity-Based Fixel Enhancement assumes that structurally connected axons are likely to undergo similar pathological changes. This assumption is used to enhance the statistical power of the analysis, mitigating the effects of correction for multiple comparisons. However, this spatial coherence assumption may also inflate results in some cases, as fixels might appear significant only due to their connection to truly affected regions, rather than real pathological changes. To some extent, this may have influenced certain findings in our analysis.
\\Additional limitations relate to the fixel-wise metrics themselves. Although these provide advantages over voxel-wise metrics by assigning values to specific fibre populations, the derived measures can still be influenced by changes in adjacent fibre populations or nonaxonal tissue compartments. These potential confounds complicate the biological interpretation as a priori knowledge about the mechanisms of the condition would be needed. For this reason, the observed increase in apparent fibre density cannot be directly interpreted as a true increase in axonal content.
\\ As the findings of this work have not previously been reported, replication in independent cohorts is necessary to confirm their robustness and further clarify the underlying biological processes.

\section{Future Developments}
Reducing the number of fixels led to increased spatial extent of significant findings but did not reveal new regions. An alternative strategy might involve a ROI-based analysis using predefined WM tracts from a template. However, given the limited knowledge about which WM tracts are most affected by chemotherapy, such a tract-restricted approach could be overly limiting. Still, focusing on WM regions related to cognition may be a valuable future direction, potentially allowing additional fixels to reach statistical significance.
\\Future work should also incorporate cognitive test scores to understand how they relate to the observed WM changes. Additionally, adopting a longitudinal study design would allow for tracking the progression of structural alterations and their relationship with cognitive outcomes over time.
\\Finally, considering the specific chemotherapy agents used could help identify differential neurotoxic effects, potentially informing more personalized treatment strategies.

Regarding the LoRE-SD method, additional work is needed to better understand the biological meaning, if any, of the metrics extracted using different scaling contrasts. In particular, applying LoRE-SD to a condition with well characterized pathological changes known to show certain effects in fixel-wise metrics using MT-CSD, could help validate and interpret these new metrics. Insights gained from such studies may provide a valuable foundation for investigating more complex conditions, such as chemobrain, where the underlying pathology is less well understood. Additionally, further exploration of scaling strategies for the ODFs, including the optimization to derive contrast matrices from existing contrasts, could help in the interpretability and robustness of LoRE-SD. 
\\Finally, in this work, scaling contrasts were designed to emphasize axonal properties, such as the intra-axonal compartment, fractional anisotropy, and white matter, allowing a direct comparison with MT-CSD, which derives metrics from the WM ODF scaled by WM volume fractions. However, the flexibility of LoRE-SD in highlighting different tissue compartments opens the possibility of developing contrasts that also capture extra-axonal contributions. Such contrasts could enable the extraction of additional metrics and potentially reveal significant effects, since chemotherapy may impact not only axons but also other tissue components. Further investigation will be required to determine whether such contrasts reveal meaningful group differences.



