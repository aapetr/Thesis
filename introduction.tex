The brain is one of the most important organs in the human body, as it is crucial for numerous functions such as cognition, motor control, language, and emotions. Due to its complexity, even minor damage can cause severe and debilitating symptoms, such as difficulties with movement, speech, or thought processing.
Brain damage may result from traumatic injuries or neurodegenerative diseases like multiple sclerosis or Alzheimer's disease, which cause gradual degeneration and death of brain tissue, leading to a progressive loss of cognitive and physical abilities. Such diseases affect millions of people worldwide, representing a major public health concern. A recent study \cite{Steinmetz2024} reported that more than 3 billion people worldwide live with a neurological condition, making it the main cause of disease and disability in the world.
\\The reason why even small injuries can have such profound effects lies in how the brain operates. Every movement, sensation, and thought relies on precise communication between specialized brain regions. These regions interact as part of a vast and intricate network, integrating sensory input, memory, and decision-making to guide everything from athletic performance to solving complex problems. This continuous integration is mediated by a tissue called white matter. 
This tissue is formed by many microscopic components, the axons, which are able to generate and propagate electrical signals, the language that our brain uses to share information. Damage to this tissue disrupts information flow and is often responsible for the symptoms seen in neurological conditions.
\\Studying the brain, however, poses a major challenge: its complex structure is protected by multiple layers, making it hidden from view and infeasible to directly access it without causing harm to the patient.
Fortunately, a non-invasive technique exists to observe and characterize white matter: diffusion-weighted magnetic resonance imaging (dMRI). This method enables researchers not only to map the brain's connectivity, but also to study how factors like age, behavior, or disease affect white matter. It works by tracking the movement of water molecules in tissues, allowing us to infer the organization of white matter tracts and to quantify their integrity. In particular, dMRI has become a powerful tool for understanding how neurodegenerative conditions impact brain structure, offering insights that could facilitate diagnosis and treatment development. 
By applying computational models to dMRI data, we can reconstruct white matter pathways and extract information about their characteristics. This way we can gain insights into how efficiently information can be transferred between brain regions, an ability that may be compromised due to disease.
\\In the next chapter, we will explore what makes white matter unique and why dMRI is particularly suited for studying it. We will also discuss the physical principles and processing methods of dMRI. Before that, however, we present a clinical application where dMRI proves valuable: chemobrain.

\vspace{1\baselineskip}
Cancer treatment has greatly improved in recent years, allowing an increasing number of patients to recover. However, with a growing population of survivors, it becomes relevant to investigate the impact of the therapies on patients' quality of life.
Recent studies have highlighted that a significant percentage of patients experience cognitive impairments in the months following chemotherapy \cite{Matsos2017}. This phenomenon is often referred to as chemobrain, with symptoms ranging from difficulties in concentration to memory loss, as shown by the results of neuropsychological tests and self-reported questionnaires, with incidence estimates ranging between 15\% and 70\% \cite{Meyers2008}. Symptoms can emerge shortly after the start of chemotherapy and persist long after its end. While these cognitive impairments were initially attributed to stress and depression following a cancer diagnosis, there is now increasing evidence that chemotherapy itself contributes to the symptoms.

One in nine women will be diagnosed with breast cancer during their lifetime \cite{Sung2021}, and chemotherapy is among the most common treatment strategies. Chemobrain is frequent also among breast cancer survivors, who often experience impairment in memory, reaction time, information processing, and executing functions \cite{Chen2020}. These cognitive tasks rely on the rapid and efficient transmission of information across different brain regions, a function largely mediated by white matter. Therefore, disruption of white matter integrity may be the cause behind the observed impairments \cite{Deprez2011}.
This hypothesis is supported by biological evidence. Shortly after the initiation of chemotherapy, patients exhibit significantly higher levels of axonal damage markers in their blood compared to non-treated individuals \cite{Schroyen2021}. This suggests that cognitive decline may be due to impaired signal transmission between brain regions caused by axonal injury or loss.
The exact mechanisms by which chemotherapy affects axons are still not fully understood. The blood-brain barrier (BBB) is expected to protect the brain from circulating toxins; however, some chemotherapeutic agents may cross the BBB and exert direct neurotoxic effects. In addition to this, chemotherapy can induce peripheral inflammation, evidenced by elevated pro-inflammatory cytokines. These inflammatory signals may initiate a cascade that leads to neuroinflammation within the brain \cite{Schroyen2021}.

Given the involvement of white matter degeneration in chemobrain, studying the induced microstructural changes can offer valuable insight into its nature. dMRI provides a non-invasive way to assess white matter integrity. Prior research has demonstrated that changes in dMRI-derived metrics before and after chemotherapy in breast cancer patients are correlated with both cognitive performance scores and levels of neural damage biomarkers in the blood \cite{Deprez2011, Schroyen2021}. However, these studies present certain limitations, including small sample sizes and limitations intrinsic to the methods used. As such, further investigation is needed to more fully characterize the chemobrain.

Within this context, the aim of the thesis is to investigate which regions are affected by chemobrain in breast cancer patients by performing a group analysis to assess local differences. State-of-the-art methods for processing dMRI data will be used and compared to a new method that might offer further flexibility in extracting microstructural information.



